\documentclass{article}
\usepackage[utf8]{inputenc}
\usepackage{amsmath}
\usepackage{graphicx}
\usepackage{hyperref}
\usepackage{geometry}
\geometry{a4paper, margin=1in}

\title{Проект: Анализ стратегий распределения защиты активов}
\author{Ваше имя}
\date{\today}

\begin{document}

\maketitle

\section*{Описание проекта}

Этот проект представляет собой инструмент для анализа различных стратегий распределения защиты активов в организации. Основная цель — максимизировать потенциальную прибыль, учитывая влияние вредоносных атак на группы активов. Проект реализован на Python с использованием библиотеки PyQt5 для графического интерфейса и \texttt{matplotlib} для визуализации данных.

\section*{Описание задачи}

Организация имеет несколько групп активов, каждая из которых приносит прибыль в зависимости от периода времени. Для защиты активов от вредоносных атак используется обновленная защита, которая может быть применена к определенным группам в определенные периоды времени. Коэффициент \( \chi_i \) определяет, насколько уменьшается прибыль группы активов при атаке, если она не защищена.

Цель проекта — найти оптимальное распределение защиты (перестановку \( \sigma \)), которое максимизирует целевую функцию \( S_1 \), представляющую собой общую прибыль с учетом защиты.

\section*{Логика реализации}

\subsection*{1. Матрица прибыли \( C \)}

Матрица \( C \) представляет собой квадратную матрицу порядка \( n \), где \( c_{ij} \) — это потенциальная прибыль \( i \)-ой группы активов за \( j \)-ый период времени при 100\% защите.

\subsection*{2. Вектор коэффициентов \( x \)}

Вектор \( x \) содержит коэффициенты \( \chi_i \), которые определяют, насколько уменьшается прибыль группы активов при атаке, если она не защищена. Каждый элемент \( x_i \) находится в диапазоне от 0 до 1.

\subsection*{3. Матрица \( D \)}

Матрица \( D \) рассчитывается на основе матрицы \( C \) и вектора \( x \). Элементы матрицы \( D \) определяются по формуле:

\[
d_{ij} = \sum_{s \in \Sigma(j-1)} (1 - \chi_s)c_{sj} + (1 - \chi_i)c_{ij} + \sum_{s=1}^n \chi_s c_{sj},
\]

где \( \Sigma(j-1) \) — множество групп, которые уже защищены к \( j \)-му периоду.

\subsection*{4. Матрица \( G \) с тильдой}

Матрица \( G \) с тильдой используется для расчета целевой функции \( S_3 \). Элементы матрицы \( G \) определяются по формуле:

\[
\widetilde{g}_{ij} = \sum_{s=j}^{n} (1 - \chi_i)c_{is}.
\]

\subsection*{5. Целевые функции}

\begin{itemize}
    \item \( S_1 \): Общая прибыль с учетом защиты.
    \item \( S_2 \): Прибыль от групп с обновленной защитой.
    \item \( S_3 \): Прибыль от групп с улучшенной защитой (оптимизируется венгерским алгоритмом).
\end{itemize}

\subsection*{6. Стратегии}

Реализованы следующие стратегии:
\begin{itemize}
    \item \textbf{Жадная стратегия}: Выбирает группу с максимальной прибылью на каждом шаге.
    \item \textbf{Минимальная стратегия}: Выбирает группу с минимальной прибылью на каждом шаге.
    \item \textbf{Максимальная стратегия}: Выбирает группу с максимальной прибылью на каждом шаге.
    \item \textbf{Случайная стратегия}: Случайно выбирает группу.
    \item \textbf{Венгерский алгоритм}: Оптимизирует распределение защиты для максимизации \( S_3 \).
\end{itemize}

\section*{Реализованные функции}

\subsection*{1. Генерация данных}
\begin{itemize}
    \item \textbf{Матрица \( C \)}: Генерируется в зависимости от выбранного режима (\texttt{random}, \texttt{increasing}, \texttt{decreasing}).
    \item \textbf{Вектор \( x \)}: Генерируется случайным образом.
\end{itemize}

\subsection*{2. Расчет матриц}
\begin{itemize}
    \item \textbf{Матрица \( D \)}: Рассчитывается на основе матрицы \( C \) и вектора \( x \).
    \item \textbf{Матрица \( G \) с тильдой}: Рассчитывается для оптимизации венгерским алгоритмом.
\end{itemize}

\subsection*{3. Стратегии}
Реализованы жадная, минимальная, максимальная, случайная стратегии и венгерский алгоритм.

\subsection*{4. Целевые функции}
Рассчитываются \( S_1 \), \( S_2 \) и \( S_3 \) для каждой стратегии.

\subsection*{5. Визуализация}
\begin{itemize}
    \item \textbf{Таблица результатов}: Отображает назначения, общую прибыль, прибыль от обновленной защиты и потери для каждой стратегии.
    \item \textbf{График потерь}: Отображает потери для каждой стратегии относительно венгерского алгоритма. График автоматически масштабируется в зависимости от диапазона данных.
\end{itemize}

\section*{Как использовать}

\subsection*{1. Запуск программы}
\begin{itemize}
    \item Убедитесь, что установлены все зависимости:
    \begin{verbatim}
    pip install numpy scipy matplotlib PyQt5
    \end{verbatim}
    \item Запустите программу:
    \begin{verbatim}
    python ui.py
    \end{verbatim}
\end{itemize}

\subsection*{2. Интерфейс}
\begin{itemize}
    \item Введите размер матрицы \( n \).
    \item Выберите режим генерации матрицы \( C \): случайный, возрастающий или убывающий.
    \item Выберите режим сортировки строк и столбцов.
    \item Нажмите \textbf{"Запустить анализ"}, чтобы рассчитать результаты.
    \item Нажмите \textbf{"Показать матрицы"}, чтобы увидеть матрицы \( C \), \( D \) и \( G \) с тильдой.
    \item Нажмите \textbf{"Показать график потерь"}, чтобы увидеть график потерь для всех стратегий.
\end{itemize}

\section*{Пример вывода}

\subsection*{Таблица результатов}

\begin{center}
\begin{tabular}{|c|c|c|c|c|}
\hline
\textbf{Стратегия} & \textbf{Назначения} & \textbf{Общая прибыль (S1)} & \textbf{Прибыль от обновленной защиты (S2)} & \textbf{Потери} \\
\hline
Жадная & [1, 2, 3] & 4800.00 & 4000.00 & 200.00 \\
Минимальная & [2, 1, 3] & 4700.00 & 3900.00 & 300.00 \\
Максимальная & [3, 2, 1] & 4900.00 & 4100.00 & 100.00 \\
Случайная & [1, 3, 2] & 4600.00 & 3800.00 & 400.00 \\
Венгерский алгоритм & [2, 3, 1] & - & - & 5000.00 \\
\hline
\end{tabular}
\end{center}

\subsection*{График потерь}

\begin{center}
\includegraphics[width=0.8\textwidth]{loss_plot.png}
\end{center}

\section*{Зависимости}

\begin{itemize}
    \item Python 3.8+
    \item Библиотеки:
    \begin{itemize}
        \item \texttt{numpy}
        \item \texttt{scipy}
        \item \texttt{matplotlib}
        \item \texttt{PyQt5}
    \end{itemize}
\end{itemize}

\section*{Авторы}

\begin{itemize}
    \item [Ваше имя]
    \item [Соавтор, если есть]
\end{itemize}

\section*{Лицензия}

Этот проект распространяется под лицензией MIT. Подробнее см. в файле \texttt{LICENSE}.

\end{document}
