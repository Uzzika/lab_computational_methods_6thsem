# Проект: Анализ стратегий распределения защиты активов

**Автор:** Ваше имя  
**Дата:** \today

---

## Описание проекта

Этот проект представляет собой инструмент для анализа различных стратегий распределения защиты активов в организации. Основная цель — максимизировать потенциальную прибыль, учитывая влияние вредоносных атак на группы активов. Проект реализован на Python с использованием библиотеки PyQt5 для графического интерфейса и `matplotlib` для визуализации данных.

---

## Описание задачи

Организация имеет несколько групп активов, каждая из которых приносит прибыль в зависимости от периода времени. Для защиты активов от вредоносных атак используется обновленная защита, которая может быть применена к определенным группам в определенные периоды времени. Коэффициент \( \chi_i \) определяет, насколько уменьшается прибыль группы активов при атаке, если она не защищена.

Цель проекта — найти оптимальное распределение защиты (перестановку \( \sigma \)), которое максимизирует целевую функцию \( S_1 \), представляющую собой общую прибыль с учетом защиты.

---

## Логика реализации

### 1. Матрица прибыли \( C \)

Матрица \( C \) представляет собой квадратную матрицу порядка \( n \), где \( c_{ij} \) — это потенциальная прибыль \( i \)-ой группы активов за \( j \)-ый период времени при 100\% защите.

### 2. Вектор коэффициентов \( x \)

Вектор \( x \) содержит коэффициенты \( \chi_i \), которые определяют, насколько уменьшается прибыль группы активов при атаке, если она не защищена. Каждый элемент \( x_i \) находится в диапазоне от 0 до 1.

### 3. Матрица \( D \)

Матрица \( D \) рассчитывается на основе матрицы \( C \) и вектора \( x \). Элементы матрицы \( D \) определяются по формуле:

$$
d_{ij} = \sum_{s \in \Sigma(j-1)} (1 - \chi_s)c_{sj} + (1 - \chi_i)c_{ij} + \sum_{s=1}^n \chi_s c_{sj},
$$

где \( \Sigma(j-1) \) — множество групп, которые уже защищены к \( j \)-му периоду.

### 4. Матрица \( G \) с тильдой

Матрица \( G \) с тильдой используется для расчета целевой функции \( S_3 \). Элементы матрицы \( G \) определяются по формуле:

$$
\widetilde{g}_{ij} = \sum_{s=j}^{n} (1 - \chi_i)c_{is}.
$$

### 5. Целевые функции

- \( S_1 \): Общая прибыль с учетом защиты.
- \( S_2 \): Прибыль от групп с обновленной защитой.
- \( S_3 \): Прибыль от групп с улучшенной защитой (оптимизируется венгерским алгоритмом).

### 6. Стратегии

Реализованы следующие стратегии:
- **Жадная стратегия**: Выбирает группу с максимальной прибылью на каждом шаге.
- **Минимальная стратегия**: Выбирает группу с минимальной прибылью на каждом шаге.
- **Максимальная стратегия**: Выбирает группу с максимальной прибылью на каждом шаге.
- **Случайная стратегия**: Случайно выбирает группу.
- **Венгерский алгоритм**: Оптимизирует распределение защиты для максимизации \( S_3 \).

---

## Реализованные функции

### 1. Генерация данных
- **Матрица \( C \)**: Генерируется в зависимости от выбранного режима (`random`, `increasing`, `decreasing`).
- **Вектор \( x \)**: Генерируется случайным образом.

### 2. Расчет матриц
- **Матрица \( D \)**: Рассчитывается на основе матрицы \( C \) и вектора \( x \).
- **Матрица \( G \) с тильдой**: Рассчитывается для оптимизации венгерским алгоритмом.

### 3. Стратегии
Реализованы жадная, минимальная, максимальная, случайная стратегии и венгерский алгоритм.

### 4. Целевые функции
Рассчитываются \( S_1 \), \( S_2 \) и \( S_3 \) для каждой стратегии.

### 5. Визуализация
- **Таблица результатов**: Отображает назначения, общую прибыль, прибыль от обновленной защиты и потери для каждой стратегии.
- **График потерь**: Отображает потери для каждой стратегии относительно венгерского алгоритма. График автоматически масштабируется в зависимости от диапазона данных.

---

## Как использовать

### 1. Запуск программы
- Убедитесь, что установлены все зависимости:
  ```bash
  pip install numpy scipy matplotlib PyQt5
